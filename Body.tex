“I believe.” We hear these words every day of our lives. No matter the context, we use these two simple words to express our thoughts about nearly everything. Whenever we want to tell others what we are thinking, or when we want to reveal the innermost affections of our hearts, we will often say, “I believe.” In His wisdom, God not only created us with the capacity to believe but also gave us an insatiable desire to explore, examine, and express our beliefs (Prov. 2; 1 Pet. 1). We possess a God-given hunger deep within our souls that causes us to examine fundamental truths about everything God has revealed to us (Deut. 4:32–40; Matt. 22:34–40).

\medskip

The mere fact that we believe in something doesn’t actually do anything for us. At the most basic level, a belief in something only provides us with the overwhelming sense that we’re not alone and that something exists beyond us. Everyone has a capacity to believe in something, and, in fact, everyone actually does believe in something (Acts 17:22–34). However, in order for belief to have heart-changing and life-changing significance, it must have the Triune God of Scripture as both its source and object (Ps. 68:26; 1 Cor. 2:5; Eph. 2:8, 9).

\medskip

Even self-proclaimed atheists actually believe God exists by virtue of what God has revealed about Himself in creation, and by virtue of the fact that all people are created in His image. We are thus left with no excuse whatsoever (Rom. 1:18–20). So-called atheists know full well there is a God—they just hate Him and find it easier for their consciences simply to pretend He does not exist. For, as we know, even demons believe God exists and rightly tremble (Mark 5:7; James 2:19).

\medskip

If everyone believes in God, the question then follows: What do we believe \textit{about} God? To answer that question is to confess, or declare, our creed. A creed is a statement that describes our beliefs. The English word creed is a cognate of the Latin word \textit{credo}, which means, “I believe.”

\medskip

Since God saw fit to give us Scripture as our only infallible guide for faith and life, it necessarily follows that Scripture is completely sufficient to serve as the final, incontrovertible judge and standard of our beliefs. Without a doubt, all we need in an absolute sense is God’s Word. That’s precisely what God Himself teaches us (John 17:17; 2 Tim. 3:16; 2 Pet. 3:16). So, then, what about the historic creeds of our faith, such as the Apostles’ Creed or the Nicene Creed? And what about all the Reformed confessions and catechisms of the sixteenth and seventeenth centuries, such as the Westminster Confession of Faith and the Heidelberg Catechism? If Scripture alone is completely sufficient for teaching, reproof, correction, and training in righteousness to the end that we would be competent and equipped for every good work, then why do we need anything else? If the Lord God Almighty wanted us to have anything beyond the sixty-six books of sacred Scripture, could He not have simply provided it to us? Are creeds and confessions really needed in the life of the Christian and in the life of the church? Why declare what the Scriptures already teach? Do not creeds and confessions simply get in the way of understanding Scripture itself?

\medskip

These are necessary and inescapable questions that every Christian must consider when it comes to creeds. And we can easily see how such questions extend not only to creeds but to the nature and purpose of the study of doctrine itself. What’s more, such questions extend naturally to any and all study of Scripture—all commentaries, all systematic theologies, all sermons, and all discussions and disputes about anything in the Bible. Anytime someone even considers for a moment what God has revealed, he has begun to formulate a creed. Whenever we sing simple songs to our children, such as “Jesus loves me, this I know, for the Bible tells me so,” we have formulated a creedal statement about Jesus, His love, the object of His love, our assurance of His love, and the nature of biblical authority.

\medskip

Significantly, the church’s historic creeds affirm that Scripture is our sole infallible and final authority. The Westminster Standards (consisting of the Westminster Confession of Faith, the Westminster Larger Catechism, and the Westminster Shorter Catechism) affirm that the sixty-six-book canon of Scripture is “given by inspiration of God to be the rule of faith and life.” Furthermore, these standards state that “the supreme judge by which all controversies of religion are to be determined, and all decrees of councils, opinions of ancient writers, doctrines of men, and private spirits, are to be examined, and in whose sentence we are to rest, can be no other but the Holy Spirit speaking in the Scripture.” The London Baptist Confession of Faith states it this way at the very outset: “The Holy Scripture is the only sufficient, certain, and infallible rule of all saving knowledge, faith, and obedience.” In essence, the church’s creeds and confessions themselves affirm that the church’s creeds and confessions of faith do not stand as authorities over Scripture but rather serve as affirmations of Scripture’s authority for all of faith and life. R.C. Sproul explains, “Creedal statements are an attempt to show a coherent and unified understanding of the whole scope of Scripture.” Creeds themselves are authoritative only in that they are subordinate to and derivative from the only divine authority, namely, the inspired and inerrant Word of God. As it is the case with pastors and individual churches, creeds cannot create new revelation, invent new teachings, or make new laws to bind the consciences of God’s people. Creeds serve to affirm the authority of God’s Word, not to stand alone as authorities unto themselves. Creeds are formulated and subscribed to as if they were theological mirrors of the Bible’s fundamental doctrines. As such, creeds exist to reflect the truth summarily, not to advance new truths. Truth isn’t created by man; it is only learned from God. Creeds merely serve to reflect and affirm, by way of systematic summary, the unchanging truth of God for the people of God.

\medskip

In fact, we find creeds even within Scripture itself. During the first century, as the apostles and other elders were forced to combat false teaching that strayed from sound doctrine, the church found it necessary to have creedal statements in order to guard against outright heresy. The apostle Paul provides us with many examples of such creeds. In many ways, these creeds are like biblically sown seeds in the rich soil of the early church that have grown into the more advanced, though fallible, creedal formulations of the church in subsequent centuries. Among many of the creeds of the apostle Paul (Rom. 10:9; 1 Cor. 15:3–7; Phil. 2:6–11; Col. 1:15–20), his words in his first letter to Timothy form perhaps the best example of a biblical creed for our purpose here:

\medskip

\begin{quote}
    
Great indeed, we confess, is the mystery of godliness:
\begin{center}
      He was manifested in the flesh,\\
         vindicated by the Spirit,\\
           seen by angels,\\
      proclaimed among the nations,\\
         believed on in the world,\\
           taken up in glory.
\end{center}
\begin{flushright}
1 Timothy 3:16 ESV
\end{flushright}
\end{quote}

The concise nature of Paul’s words of confession to Timothy serve as a good example of something we see throughout all of Scripture; for example, the Ten Commandments (Ex. 20:1–21), the \textit{Shema} (Deut. 6:4, 5), John’s summary of the gospel (John 3:16), and Paul’s formulation of his gospel message (1 Cor. 15:3–7). Throughout the apostolic era and afterward, as the effects of sin continued to yield false teachers and their false teaching, and as new heresies were invented, the people of God found it necessary and prudent to reaffirm time and again the sound doctrine of Scripture. Much like the church’s creeds through the centuries, the creeds of the apostolic era and the early church were formulated to serve the church as God spiritually matured His people and advanced His kingdom throughout the world.

\medskip

So, whether we have in mind the only inspired and infallible creedal formulations in Scripture itself, or uninspired and fallible creedal formulations—such as the Apostles’ Creed, the Nicene Creed, the London Baptist Confession of Faith, the Three Forms of Unity of the Dutch Reformed churches (the Heidelberg Catechism, the Canons of Dort, and the Belgic Confession), and the Westminster Confession of Faith and Catechisms—it is crucial that we understand the church’s God-given duty to be a faithful steward and guardian of the one and only faith once delivered to the saints. We must steward and guard our faith in order to provide the church of all generations with carefully worded, concise summaries of the doctrine of Scripture.

\medskip

While there are certainly differences among us as Christians, it is because of our stated beliefs—our formal, written creeds and confessions—that we are able to maintain unity grounded in the essential matters of salvation and show liberty on matters not essential to the church’s fundamental doctrines, all the while maintaining our gospel bond with biblically informed and truth-motivated love, recognizing the importance of each and every matter addressed in God’s Word. What is so amazing about the church’s historic creeds is not their doctrinal differences, which certainly do exist, but the overwhelming doctrinal agreement among them. Indeed, the unity of the church is proven not through doctrinal compromise, declared peace, or perceived unity in spite of our secondary and tertiary doctrinal differences, even though all our differences are important. Rather, the unity of the church is proven in our affirmation, confession, and proclamation of the fundamental doctrinal matters of our one Christian religion, without which unity we do not stand within the church but remain outside the one and only church of God.

\medskip

The church’s creeds and confessions exist, like our very lives, to glorify God according to His truth and, thus, to enjoy Him forever. We do this by believing, confessing, and proclaiming our doctrine, piety, and practice in accordance with what He has revealed and not according to the superstitions of men, the deceitful schemes of Satan, or the arrogant and presumptuous notions of our own hearts. It may be helpful to think of creeds as maps or guides to help us navigate our way as we study God’s Word, looking to the doctrinal keys formulated by our forefathers. While someone could argue that we don’t really need maps in order to travel, we all know how helpful maps are if we want to arrive at a particular destination on a particular route in a particular amount of time. A good map is the result, most often, of many qualified people devoting a significant amount of time, energy, and wisdom in order to ensure the map’s quality and accuracy. And, like a good map, a good creed or confession is the result of a significant amount of time, energy, and wisdom devoted by our forefathers to mapping the doctrinal contours of Scripture. The Bible is a beautiful and vast world of mountains, rivers, and paths, and we are called to climb them, navigate them, and walk them as we look to, learn from, and lean on those who have traveled them faithfully in generations past.

\medskip

Throughout the ages creeds have come under attack on numerous occasions, and such attacks have come almost exclusively from heretics outside the church, which, incidentally, ought to tell us something. Even today at the beginning of the twenty-first century there seems to be a disregard, albeit even disdain, for creeds, not to mention a growing ignorance of creeds even among those who profess to subscribe to them. With this in mind, I offer an apologetic of sorts for the formulation, usefulness, and purpose of the church’s creeds and confessions. Though by no means exhaustive, I offer this list to provide the church with a summary of the purpose of creeds, in no particular order. The purpose of creeds is:

\medskip

1. To \textit{glorify} God according to his truth and to enjoy him forever by believing, confessing, and proclaiming our doctrine in accordance with what he has revealed and not according to the superstitions of men, the deceitful schemes of Satan, or the arrogant and presumptuous notions of our own hearts.

\medskip

2. To \textit{affirm} the one true God almighty who has revealed himself to us and whose glorious attributes, gracious laws, and grand story of redemption point us to himself as our only Lord to the end that we might love him rightly and as fully as possible with all our heart, with all our soul, with all our mind, and with all our strength.

\medskip

3. To \textit{guard} the unchanging, sound doctrine of Scripture against false teachers and heretics outside the church, and to guard against the vain and false notions of Scripture from within the church as a shining witness of God’s truth to the watching world out of which God calls his elect through the preaching of the gospel and inward call of the Holy Spirit.

\medskip

4. To \textit{discern} truth from doctrinal error and to discern truth from half- truth as we contend earnestly for the faith once delivered to the saints that we might grow up in every way into Christ, who is the living head of the church, who is the way, the truth, the life, and the only way to the Father.

\medskip

5. To \textit{remain} steadfast through the ages until Christ’s return as one, holy, catholic, and apostolic church of Christ who believe, confess, and proclaim the pure and unadulterated Word of God and who rightly administer the sacraments of baptism and the Lord’s Supper, including our consistent exercise of church admonition, correction, and discipline.

\medskip

6. To \textit{uphold} the life-encompassing doctrine of the inspired and inerrant Word of God as our sole, infallible authority that is profitable for doctrine, for reproof, for correction, for instruction in righteousness to the end that every man of God might be complete, thoroughly equipped for every good work.

\medskip

7. To \textit{maintain} freedom for individual Christians as well as the entire church from extra-biblical laws, traditions, and superstitions of men that bind men’s consciences, perplex men’s souls, lead our children astray according to their sin, and bring about man-exalting pride instead of God-exalting humility.

\medskip

8. To \textit{confirm} men according to the church’s doctrinal standard who have been elected to serve as officers of the church as well as to equip, examine, and prove those men who have been called as pastors and elders over the flock of God, and to ascertain their suitability to teach as they feed, care for, and pray with and for the sheep of Christ for whom he gave his life.

\medskip

9. To \textit{preserve} the purity and, thereby, the peace and unity of the church visible as the outward witness of Christ and his elect bride, the church invisible, to the end that we might stand together as one family with one Father, one Lord, one faith, one baptism, unwaveringly according to and because of the truth, never in spite of, disregard for, or ignorance of it.

\medskip

10. To \textit{fulfill} the Great Commission in our united affirmation and proclamation of the one true gospel of Jesus Christ, which is the only power of God unto salvation to all who believe, by making disciples in our homes, churches, communities, and in all nations, baptizing them in the name of the Father, Son, and Holy Spirit and teaching them to observe all things that our Lord Jesus Christ commanded us.

\medskip

It is for the sake of our love for each other, for the body of Christ as the one, holy, catholic, and apostolic church, and for our love of Christ Himself that we must affirm, confess, and proclaim the unchanging doctrine of Scripture summarized in our historic, orthodox creeds.